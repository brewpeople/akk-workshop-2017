\documentclass[18pt]{beamer}
\usetheme[progressbar=frametitle]{metropolis}
\usecolortheme[snowy]{owl}

\usepackage{ngerman}
\usepackage{csquotes}

\title{Der große Bierbrauworkshop}
\author{foo}

\begin{document}
\maketitle

% \section{Überblick}

\begin{frame}{Vorstellung}
  \begin{itemize}
    \item Max
    \item Matze
    \item brauen seit etwa 2012 (?)
  \end{itemize}
  \url{bier@lists.kit.edu}
  \url{noerdbier.de}
  \url{github.com/brewpeople}
\end{frame}
\begin{frame}{Technisches}
  \begin{itemize}
    \item Brausteuerung
    \item brewpeople
  \end{itemize}
\end{frame}
\begin{frame}{Geschichtlicher Ursprung}
  Irgendwas mit Ägypten
\end{frame}
\begin{frame}{Reinheitsgebot}
  \begin{block}{Ursprung}
    \begin{itemize}
      \item Marketingbegriff aus dem 20. Jahrhundert
      \item Bezug auf eine bayerische Vorschrift von 1516
    \end{itemize}
  \end{block}
  \begin{block}{Bedeutung}
    \begin{itemize}
      \item Nur aus \emph{Hopfen}, \emph{Malz}, \emph{Hefe} und \emph{Wasser}
    \end{itemize}
  \end{block}
\end{frame}
\begin{frame}{Sauberkeit}
  Nach dem Hopfenkochen peinlichst drauf achten, die Würze nicht zu infizieren.
\end{frame}
\begin{frame}{Das Tagesrezept}
  \begin{itemize}
    \item Pale Ale
  \end{itemize}
\end{frame}
\begin{frame}{Kennzahlen}
  \begin{description}
    \item[IBU] International Bitter Units --- Bitterkeit
    \item[EBC] European Brewing Convention --- Bierfarbe
    \item[\textalpha-Säure]  foo
  \end{description}
\end{frame}
\begin{frame}{Bierfarbe}
  siehe Tabelle im Hobbybrauer wiki
\end{frame}
\begin{frame}{Der Brauprozess im Überblick}
  \begin{enumerate}
    \item Malzen und Schroten
    \item Einmaischen und Rasten \enquote{fahren}
    \item Läutern
    \item Hopfen kochen und Würzeklärung
    \item Vergärung und Lagerung
  \end{enumerate}
\end{frame}
\begin{frame}{Hopfen}
  \begin{block}{Warum?}
    \begin{itemize}
      \item Aroma
      \item Haltbarkeit
    \end{itemize}
  \end{block}
\end{frame}
\begin{frame}{Minimalausstattung}
  \begin{itemize}
    \item Gefäß zum Kochen
    \item Läuterbottich mit Läuterblech
    \item Gärfass
  \end{itemize}
\end{frame}

\end{document}
