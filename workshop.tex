\documentclass[18pt]{beamer}
\usetheme[progressbar=frametitle]{metropolis}
\usecolortheme[snowy]{owl}

\usepackage{ngerman}
\usepackage{csquotes}
\usepackage{booktabs}

\title{Der große Bierbrauworkshop}
\author{foo}

\begin{document}
\maketitle


% http://www.cloneabeer.com/CABebc.php
\definecolor{ebc4}{RGB}{255, 230, 153}
\definecolor{ebc8}{RGB}{255, 191, 66}
\definecolor{ebc12}{RGB}{248, 166, 0}
\definecolor{ebc20}{RGB}{222, 124, 0}
\definecolor{ebc35}{RGB}{134, 24, 0}
\definecolor{ebc61}{RGB}{90, 10, 2}
\definecolor{ebc79}{RGB}{54, 8, 10}

% \section{Überblick}

\begin{frame}{Vorstellung}
  \begin{itemize}
    \item Max
    \item Matze
    \item brauen seit etwa 2012 (?)
  \end{itemize}
  \url{bier@lists.kit.edu}
  \url{noerdbier.de}
  \url{github.com/brewpeople}
\end{frame}
\begin{frame}{Technisches}
  \begin{itemize}
    \item Brausteuerung
    \item brewpeople
  \end{itemize}
\end{frame}
\begin{frame}{Geschichtlicher Ursprung}
  Irgendwas mit Ägypten
\end{frame}
\begin{frame}{Reinheitsgebot}
  \begin{block}{Ursprung}
    \begin{itemize}
      \item Marketingbegriff aus dem 20. Jahrhundert
      \item Bezug auf eine bayerische Vorschrift von 1516
    \end{itemize}
  \end{block}
  \begin{block}{Bedeutung}
    \begin{itemize}
      \item Nur aus \emph{Hopfen}, \emph{Malz}, \emph{Hefe} und \emph{Wasser}
    \end{itemize}
  \end{block}
\end{frame}
\begin{frame}{Sauberkeit}
  Nach dem Hopfenkochen peinlichst drauf achten, die Würze nicht zu infizieren.
\end{frame}
\begin{frame}{Das Tagesrezept}
  \begin{itemize}
    \item Pale Ale
  \end{itemize}
\end{frame}
\begin{frame}{Kennzahlen}
  \begin{description}
    \item[IBU] International Bitter Units --- Bitterkeit
    \item[EBC] European Brewing Convention --- Bierfarbe
    \item[\textalpha-Säure]  foo
  \end{description}
\end{frame}
\begin{frame}{Bierfarbe}
  \tikzset{
    ebc bar/.style={
      rectangle,
      minimum width=2cm,
    }
  }
  \begin{table}
    \begin{tabular}{llll}
      \textbf{EBC} & \textbf{englisch} & \textbf{deutsch} & \textbf{Farbe}\\
      \midrule
      4--8  & pale & hell & \tikz {\node[ebc bar, left color=ebc4, right color=ebc8] {}} \\
      8--12 & golden, pale & gold & \tikz {\node[ebc bar, left color=ebc8, right color=ebc12] {}} \\
      12--20 & amber & bernstein & \tikz {\node[ebc bar, left color=ebc12, right color=ebc20] {}} \\
      20--35 & light brown, copper & kupfer & \tikz {\node[ebc bar, left color=ebc20, right color=ebc35] {}} \\
      35--60 & brown & braun & \tikz {\node[ebc bar, left color=ebc35, right color=ebc61] {}} \\
      60+ & dark brown, black & schwarz & \tikz {\node[ebc bar, left color=ebc61, right color=ebc79] {}}
    \end{tabular}
  \end{table}
\end{frame}
\begin{frame}{Untergärige Biertypen}
  \begin{table}
    \begin{tabular}{llllll}
      \textbf{Kategorie} & \textbf{StW \textdegree P} & \textbf{IBU} & \textbf{EBC} & \textbf{\% vol} \\
      \midrule
      Leichtes Lager & 7,0--13,8 & 8--30 & 4--12 & 2,8--6,0 \\
      Pils & 11,0--14,7 & 25--45 & 4--12 & 4,4--6,0 \\
      Europ. bernsteinf. Lager & 11,4--14,0 & 18--30 & 14--32 & 2,6--5,7 \\
      Dunkles Lager & 11,0--13,8 & 8--32 & 28--59 & 4,2--6,0 \\
      Bock & 15,7--28,0 & 16--35 & 12--59 & 6,3--14,0 \\
    \end{tabular}
  \end{table}
\end{frame}
\begin{frame}{Obergärige Biertypen -- Ales}
  \begin{table}
    \begin{tabular}{llllll}
      \textbf{Kategorie} & \textbf{StW \textdegree P} & \textbf{IBU} & \textbf{EBC} & \textbf{\% vol} \\
      \midrule
      English Pale Ale & 8,0--14,7 & 23--50 & 8--35 & 3,2--6,2 \\
      Scot./Irish Pale Ale & 7,6--30,1 & 10--35 & 18--49 & 2,5--10,0 \\
      American Ale & 11,2--14,7 & 20--45 & 10--69 & 4,3--6,2 \\
      English Brown Ale & 7,6--12,9 & 10--30 & 24--69 & 2,8--5,4 \\
      India Pale Ale & 12,4--21,6 & 40--120 & 12--30 & 5,0--10,0 \\
      Belg./Franz. Ale & 11,0--19,3 & 10--35 & 4--37 & 4,5--8,5 \\
      Belg. Starkbier & 15,2--25,9 & 15--40 & 6--43 & 6,0--11,0 \\
      Strong Ale & 14,7--28,0 & 30--120 & 16--43 & 6,0--12,0
    \end{tabular}
  \end{table}
\end{frame}
\begin{frame}{Obergärige Biertypen -- andere}
  \begin{table}
    \begin{tabular}{llllll}
      \textbf{Kategorie} & \textbf{StW \textdegree P} & \textbf{IBU} & \textbf{EBC} & \textbf{\% vol} \\
      \midrule
      Helles Hybrid & 9,5--13,6 & 15--30 & 5--12 & 2,1--5,6 \\
      Bernsteinf. Hybrid & 11,4--13,3 & 25--50 & 20--33 & 4,5--5,5 \\
      Porter & 10,0--21,6 & 18--50 & 33--69 & 4,5--9,5 \\
      Stout & 9,0--27,0 & 25--90 & 43--79 & 4,0--12,0 \\
      Weizen- u. Roggenbier & 11,0--21,6 & 8--30 & 4--49 & 4,3--8,0 \\
      Sauerbier & 7,1--18,0 & 0--25 & 4--43 & 2,8--8,0
    \end{tabular}
  \end{table}
\end{frame}
\begin{frame}{Der Brauprozess im Überblick}
  \begin{enumerate}
    \item Malzen und Schroten
    \item Einmaischen und Rasten \enquote{fahren}
    \item Läutern
    \item Hopfen kochen und Würzeklärung
    \item Vergärung und Lagerung
  \end{enumerate}
\end{frame}
\begin{frame}{Hopfen}
  \begin{block}{Warum?}
    \begin{itemize}
      \item Aroma
      \item Haltbarkeit
    \end{itemize}
  \end{block}
\end{frame}
\begin{frame}{Minimalausstattung}
  \begin{itemize}
    \item Gefäß zum Kochen
    \item Läuterbottich mit Läuterblech
    \item Gärfass
  \end{itemize}
\end{frame}

\end{document}
